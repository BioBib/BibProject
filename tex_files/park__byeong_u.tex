% arara: pdflatex
% arara: biber
% arara: pdflatex
% arara: pdflatex
\documentclass[letterpaper]{article}
\usepackage{imscv}
\addbibresource{imsBooks.bib}
\addbibresource{\jobname.bib}
\usepackage{filecontents}
\begin{filecontents}{\jobname.bib}
%% BibTeX file for Park, Byeong U.
\end{filecontents}

\begin{document}

\Name{Park, Byeong U.}
\ID{park__byeong_u}
\AltID{park__byeong}



\subsection*{Profiles}
\Profile{http://www.ams.org/mathscinet/search/author.html?mrauthid=136225}{MathSciNet}
\Profile{http://zbmath.org/authors/?q=ai:park.byeong-uk}{zbMATH}
\Profile{http://genealogy.math.ndsu.nodak.edu/id.php?id=34342}{Math. Genealogy}
\Profile{http://academic.research.microsoft.com/Author/10712892}{Microsoft Academic: Park, Byeong Uk}


\subsection*{Education}
\Degree{1987}{Ph.D.}{University of California, Berkeley}[Efficient Estimation in the Two-Sample Semiparametric Location Scale Model and the Orientation Shift Model]<http://genealogy.math.ndsu.nodak.edu/id.php?id=34342>


\subsection*{Honors}
\Honor{2005}{IMS Fellow}{For seminal work in bandwidth selection for kernel smoothing, which led to a large, fruitful and well-cited body of work in that area; for his leadership research in semiparametrics, smoothing, additive and single index models, stochastic frontier estimation, and asymptotic analysis in general; for a leading role in Korean statistics, including valued editorial work; for strong mentoring of, and inspiration of, a generation of Korea statisticians}
\end{document}