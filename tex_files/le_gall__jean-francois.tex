% arara: pdflatex
% arara: biber
% arara: pdflatex
% arara: pdflatex
\documentclass[letterpaper]{article}
\usepackage{imscv}
\addbibresource{imsBooks.bib}
\addbibresource{\jobname.bib}
\usepackage{filecontents}
\begin{filecontents}{\jobname.bib}
%% BibTeX file for Le Gall, Jean-François
\end{filecontents}

\begin{document}

\Name{Le Gall, Jean-Fran\c{c}ois}
\ID{le_gall__jean-francois}




\Homepage{http://www.cmi.univ-mrs.fr/~pardoux/Ecole_doctorale/Cours_Le%20Gall.htm}


\subsection*{Profiles}
\Profile{http://www.ams.org/mathscinet/search/author.html?mrauthid=111380}{MathSciNet}
\Profile{http://zbmath.org/authors/?q=ai:le-gall.jean-francois}{zbMATH}
\Profile{http://genealogy.math.ndsu.nodak.edu/id.php?id=80792}{Math. Genealogy}
\Profile{http://academic.research.microsoft.com/Author/1121967}{Microsoft Academic: Jean-Francois Le Gall (Jean-Fran\c{c}ois Le Gall)}


\subsection*{Education}
\Degree{year?}{Ph.D.}{}<http://genealogy.math.ndsu.nodak.edu/id.php?id=80792>


\subsection*{Honors}
\Honor{1986}{Rollo Davidson Prize}
\Honor{2008}{IMS Fellow}{For contributions to the fine properties of Brownian motion and to superprocesses, in particular, for his invention of the Brownian snake and its applications to the study of the sample path properties of super Brownian motion and to the resolution of conjectures for non-linear partial differential equations}
\Honor{2010}{Wald Lecturer}
\end{document}